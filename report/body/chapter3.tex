
\section{实验环境与平台}

本次试验的硬件信息如下\ref{table3-1},在我的个人笔记本上进行训练,表格,包括设备信息,设备配置等。

\begin{table}[h]
    \caption{\textbf{硬件信息}}
    \label{table3-1}
    \centering
    \begin{tabularx}{\linewidth}{>{\centering}p{3.4cm} >{\centering\arraybackslash}X}
        \toprule 
        \textbf{项目}   &   \textbf{信息} \\
        \midrule
        设备名称     &   82RC Legion Y7000P IAH7 \\
        CPU         &   12th Gen Intel i7-12700H (20) @ 4.600GHz \\
        GPU         &   NVIDIA Corporation GA107BM [GeForce RTX 3050 Ti Mobile] \\
        RAM         &   16GB \\
        系统类型     &   x86\_64 \\
        \bottomrule
    \end{tabularx}
\end{table}

\vspace*{2cm}

实验机系统在linux,表格\ref{table3-2}包括系统信息,版本信息等。

\begin{table}[h]
    \caption{\textbf{软件信息}}
    \label{table3-2}
    \centering
    \begin{tabularx}{\linewidth}{>{\centering}p{3.4cm} >{\centering\arraybackslash}X}
        \toprule 
        \textbf{项目}       &   \textbf{信息} \\
        \midrule 
        操作系统类型         &   Arch Linux x86\_64 \\
        系统版本            &  6.9.1-arch1-1 \\
        cuda版本            &   12.4 \\
        python版本          &   3.11.7 \\
        numpy版本           &   1.26.3 \\
        pytorch版本         &   2.20(cuda ver.) \\
        scikit\_learn版本    &   1.4.2 \\
        \bottomrule
    \end{tabularx}
\end{table}


